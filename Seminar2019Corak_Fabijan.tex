\documentclass[times, utf8, seminar, numeric]{fer}
\usepackage{booktabs}
\usepackage{url}

\begin{document}

% Ukljuci literaturu u seminar
\nocite{*}

\title{Automatsko generiranje pitanje}

\author{Fabijan Čorak}

\mentor{izv.\ prof.\ dr.\ sc.\ Jan Šnajder}

\maketitle

\tableofcontents

\chapter{Uvod}
Ovaj seminarski rad predstavlja pregled područja automatskog generiranja pitanja. Riječ je o postupku ekstrakcije pitanja iz nekog polaznog teksta bez aktivnog ljudskog posredovanja. Kome ovo može koristiti? Nastavnicima u evaluaciji studenata, kao i samim studentima u provjeravanju njihovog znanja \cite{labutov2015deep}.

Ovisno o veličini polaznog teksta rješenja na ovom području mogu se podijeliti na pristupe \textit{izjavna rečenica} - \textit{upitna rečenica} i \textit{paragraf} - \textit{upitna rečenica}. Pravi je izazov metoda koja jedno provodi u drugo. Osvrt je načinjen kroz analizu triju različitih pristupa opisanih u radovima \cite{du2018harvesting}, \cite{flor2018semantic} i \cite{labutov2015deep}. Za svaki rad opisana je njegova ideja kao slijed koraka od problema do rješenja te postupak vrednovanja. Dodatno, komentirani su prednosti odnosno nedostaci svakog pristupa.

\chapter{Duboka pitanja bez dubokog razumijevanja}
\section{Ideja}
Naslovu unatoč, pristup opisan u radu \cite{labutov2015deep} nema veze s dubokim učenjem. Pitanja su duboka jer proizlaze iz čitavog predočenog teksta, a bez razumijevanja su jer nastaju prema generičnim predlošcima s kojima nisu u direktnoj svezi. 

\chapter{Zaključak}
Zaključak.

\bibliography{literatura}
\bibliographystyle{fer}

\chapter{Sažetak}
Sažetak.

\end{document}
