\documentclass[11pt]{beamer}
\usepackage[utf8]{inputenc}
\usepackage[english]{babel}
\usepackage{array}
\usetheme{Darmstadt}
\graphicspath{{../img/}}
\begin{document}
	\author{Fabijan Čorak \\ \small{mentor: izv.\ prof.\ dr.\ sc.\ Jan Šnajder}}
	\title{Generiranje pitanja pomoću strojnog učenja}
	%\subtitle{}
	%\logo{}
	\institute{Fakultet elektrotehnike i računarstva \\ Sveučilište u Zagrebu}
	\date{\tiny 5.\ lipnja 2019.}
	%\subject{}
	%\setbeamercovered{transparent}
	%\setbeamertemplate{navigation symbols}{}
	\begin{frame}[plain]
	\maketitle
\end{frame}

\begin{frame}
\frametitle{Uvod}
\begin{itemize}
	\item engleski jezik
	\item domenska neovisnost
\end{itemize}
\end{frame}

%\begin{frame}
%\frametitle{Uvod}
%Dva pristupa
%\begin{itemize}
%	\item jedna izjavna rečenica $\rightarrow$ upitna rečenica
%	\item odlomak $\rightarrow$ upitna rečenica
%\end{itemize}
%\end{frame}

\begin{frame}
\frametitle{Duboka pitanja bez dubokog razumijevanja}
\begin{itemize}
	\item Labutov et al., 2015.
	\item dorađena Freebase kategorizacija svih wiki-članaka\\
	(\textit{osoba, mjesto, događaj, organizacija, ...})
	\item dodatno: svi članci već su podijeljeni na sekcije\\
	(\textit{rani život, demografija, povijest, ...})
	\item ontologija parova (kategorija, sekcija)\\
	(\textit{(osoba, rani život), (osoba, politička karijera),\\..., (mjesto, demografija), ...})
\end{itemize}
\end{frame}

\begin{frame}
\frametitle{Duboka pitanja bez dubokog razumijevanja}
\begin{itemize}
	\item \textit{crowdsourcing} predložaka pitanja za svaki par u ontologiji,\\
	npr.\ \textit{(osoba, rani život)}\\
	Who were the key influences on \_\_\_\_\_\_\_ in their childhood?
	\item Amazon Mechanical Turk
\end{itemize}
\end{frame}

\begin{frame}
\frametitle{Duboka pitanja bez dubokog razumijevanja}
Uporaba
\begin{itemize}
	\item klasifikacija kategorije, sekcije ulaznog teksta\\
	LR (200 000 tf-idf)
	\item dohvaćanje predložaka iz baze za utvrđeni par (kategorija, sekcija)
	\item klasifikacija prikladnosti predloška za pitanje\\
	LR (200 000 tf-idf + 300 word2vec)
\end{itemize}
\end{frame}

\begin{frame}
\frametitle{Duboka pitanja bez dubokog razumijevanja}
Metrika - IR
\begin{itemize}
	\item ljudsko vrednovanje ekstrakcije pitanja na tekstovima iz 12 wiki-članaka
	\item za svaki ulazni segment teksta sustav: izlaz - pitanja poredana prema procijenjenoj razini prikladnosti
	\item kretanje preciznosti za različite razine odziva
\end{itemize}
\end{frame}

\begin{frame}
\frametitle{Duboka pitanja bez dubokog razumijevanja}
	\begin{itemize}
		\item duboko učenje?
		\item duboka pitanja - temeljena na većem tekstu (više od jedne rečenice)
		\item bez dubokog razumijevanja - sustav ne zna odgovore na pitanje
	\end{itemize}
\end{frame}

\begin{frame}
\frametitle{Pristup automatskom generiranju domenski neovisnih pitanja temeljen na semantičkim ulogama}
\begin{itemize}
	\item Flor et Riordan, 2018.
	\item sustav temeljen na pravilima
	\item alati temeljeni na strojnom učenju za POS-tagging i SRL
	\item SRL - Propbank notacija
\end{itemize}
\end{frame}

\begin{frame}
	\begin{table}[h]
		\begin{center}
			\begin{tabular}{ |c|c| } 
				\hline
				\textbf{Oznaka} & \textbf{Uloga}\\
				\hline
				A0 & proto-agent (često gramatički subjekt)\\
				\hline
				A1 & proto-pacijent (često gramatički objekt)\\
				\hline
				A2 & instrument, atribut, primanje, količina itd.\\
				\hline
				A3 & početna točka ili stanje\\
				\hline
				A4 & krajnja točka ili stanje\\
				\hline
				AM-LOC & mjesto\\
				\hline
				AM-DIR & smjer\\
				\hline
				AM-TMP & vrijeme\\
				\hline
				AM-CAU & uzrok\\
				\hline
				AM-PNC & svrha\\
				\hline
				AM-MNR & način\\
				\hline
				AM-EXT & doseg\\
				\hline
				AM-DIS & izlaganje\\
				\hline
				AM-ADV & prilog\\
				\hline
				AM-MOD & modalni glagol\\
				\hline
				AM-NEG & negacija\\
				\hline
			\end{tabular}
		\end{center}
	\end{table}
\end{frame}

\begin{frame}
\frametitle{Pristup automatskom generiranju domenski neovisnih pitanja temeljen na semantičkim ulogama}
Uporaba
\begin{itemize}
	\item za svaku rečenicu ulaznog teksta: da/ne-pitanje i za skoro svaku semantičku ulogu u rečenici wh-pitanje
	\item na temelju podatka o POS i SRL odabire se upitna (wh) riječ (\textit{where, when, how, ...})
	\item u samom procesu pretvorbe izjavne u upitnu rečenicu korištena su jezična pravila\\
	(njihov točan oblik nije naveden u radu)
\end{itemize}
\end{frame}

\begin{frame}
\frametitle{Pristup automatskom generiranju domenski neovisnih pitanja temeljen na semantičkim ulogama}
Metrika
\begin{itemize}
	\item ljudsko vrednovanje izlaza
	\item gramatička točnost (1-5)
	\item semantička točnost\\
	(koliko dobro pitanje razumije značenje izvorne rečcenice; 1-5)
	\item relevantnost\\
	(koliko je pitanje vezano uz informacije u izvornoj rečenici; 0-3)
\end{itemize}
\end{frame}

\begin{frame}
\frametitle{Pristup automatskom generiranju domenski neovisnih pitanja temeljen na semantičkim ulogama}
Primjeri
\begin{itemize}
	\item \textit{The particles from the Sun also carry an electric charge.}\\
	$\rightarrow$ \textit{Do the particles from the Sun carry an electric charge?}\\
	(ocjene: 5, 5, 3)
	\item \textit{Deep below is a place called the magma chamber.}\\
	$\rightarrow$ \textit{Did a place call the magma chamber?}\\
	(ocjene: 4.5, 1.5, 2.5)
\end{itemize}
\end{frame}

\begin{frame}
\frametitle{Prikupljanje parova pitanja i odgovora temeljenih na odlomcima s Wikipedije}
\begin{itemize}
	\item Du et Cardie, 2018.
	\item duboki pristup! (BiLSTM encoder-decoder + attention)
	\item kao i prethodni rad: rečenica $\rightarrow$ upit\\
	+ rečenica se obogaćuje kontekstualnim znanjem (umetanje koreferentnih tokena)
	\item određivanje koreferentnih tokena - gotov model
	\item SQuAD
\end{itemize}
\end{frame}

\begin{frame}
\frametitle{Prikupljanje parova pitanja i odgovora temeljenih na odlomcima s Wikipedije}
Identifikacija odgovora
\begin{itemize}
	\item prvi zadatak sustava
	\item samo jedan odgovor (ograničavajuće)
\end{itemize}
\end{frame}

\begin{frame}
%\frametitle{Prikupljanje parova pitanja i odgovora temeljenih na odlomcima s Wikipedije}
\begin{figure}
	\centering
	\hbox{\hspace{-2.5em} \includegraphics[width=12.5cm]{gated_coref_nqg.png}}
\end{figure}
\begin{table}[h]
	\tiny
	\hbox{\hspace{-4em} \begin{tabular}{|c|ccccccccccc|}
		\hline
		word & they & the & panthers & defeated & the & arizona & cardinals & 49 & - & 15 & ...\\
		ans. feature & O & O & O & O & B\_ANS & I\_ANS & I\_ANS & O & O & O & ...\\
		coref.feature & B\_PRO & B\_ANT & I\_ANT & O & O & O & O & O & O & O & ...\\
		\hline
	\end{tabular}}
\end{table}
\end{frame}

\begin{frame}
\frametitle{Prikupljanje parova pitanja i odgovora temeljenih na odlomcima s Wikipedije}

\begin{itemize}
	\item \textit{word embedding} + značajka odgovora + značajka koreferentnosti
	\item profinjenje BIO-tagginga koreferentnosti (inovativnost)
	\item $c_i \rightarrow d_i$ prema formuli: $d_i = c_i \odot ReLU(W_a c_i + W_b score_i + b)$
\end{itemize}
\end{frame}

\begin{frame}
\frametitle{Prikupljanje parova pitanja i odgovora temeljenih na odlomcima s Wikipedije}
Metrika
\begin{itemize}
	\item ljudsko označavanje izlaza sustava (gramatičnost + smisao + odgovorivost)
	\item metrike iz strojnog prevođenja (BLEU, METEOR)
	\item postojeći model za davanje odgovora trenira se na generiranim pitanjima i evaluira
\end{itemize}
\end{frame}

\begin{frame}
\frametitle{Prikupljanje parova pitanja i odgovora temeljenih na odlomcima s Wikipedije}
Primjeri
\begin{itemize}
	\item \textit{The elizabethan navigator, sir francis drake
		was born in the nearby town of tavistock and was the
		mayor of plymouth. ... . he died of dysentery in 1596 off the coast of puerto rico.}
	\item gold: \textit{In what year did Francis Drake die?}
	\item baseline (neprofinjeni kontekst): \textit{When did he die?}
	\item generirano: \textit{When did sir francis drake die?}
\end{itemize}
\end{frame}

\begin{frame}
\frametitle{Zaključak}

\begin{table}[h]	
	\begin{tabular}{|c|m{3cm}|m{3cm}|}
			\hline
			Pristup & Prednosti & Nedostaci \\
			\hline
			Predlošci + LR & zanimljiv pristup, primjenjiv drugdje & nepoznavanje odgovora\\
			\hline
			Pravila + SRL & neovisnost o treniranju & jedna rečenica - nerobusnost \\
			\hline
			BiLSTM + koref. & ljudski neovisne metrike & jedna rečenica - jedno pitanje\\
			\hline
	\end{tabular}
\end{table}
\end{frame}

\begin{frame}
\begin{center}
	\Large{Hvala na pažnji}
\end{center}
\end{frame}

\end{document}